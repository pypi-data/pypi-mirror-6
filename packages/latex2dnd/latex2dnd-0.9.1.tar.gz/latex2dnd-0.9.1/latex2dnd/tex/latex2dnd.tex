%
% latex2dnd tex code
%
% use %
% latex2dnd tex code
%
% use %
% latex2dnd tex code
%
% use %
% latex2dnd tex code
%
% use \input{latex2dnd} to include this at the top of your tex file
%

\usepackage{zref-user}
\usepackage{zref-abspos}
\usepackage{tikz}
\usepackage{forloop}
\usepackage{calc}
\usepackage{newfile}
\usepackage{ifthen}

\newoutputstream{dnd}
\openoutputfile{\jobname.dnd}{dnd}

\newcounter{BoxCount}
\setcounter{BoxCount}{0}
\newcounter{I}

\pagestyle{empty}
\usetikzlibrary{calc}

\newcommand\DDlabel[2]{%
  % #1 = label name
  % #2 = label contents, e.g. math symbols or word
  \stepcounter{BoxCount}%
  \expandafter\newsavebox\expandafter{\csname BOX\alph{BoxCount}\endcsname}%
  \expandafter\sbox\expandafter{\csname BOX\alph{BoxCount}\endcsname}{{#2}}%
  \expandafter\edef\csname BOXName#1\endcsname{\csname BOX\alph{BoxCount}\endcsname}%
  \addtostream{dnd}{LABEL: \arabic{BoxCount} = #1}%
}%

\newcommand{\writeDDlabels}[1][4.3ex]{
  % write out drag-drop labels
  % arguments:
  %
  % #1 = box height (optional, defaults to 4.3ex)
  \newpage
  TOP \par
  \stepcounter{BoxCount}%
  \setcounter{I}{1}%
  \forloop{I}{1}{\value{I} < \value{BoxCount}}{%
    \typeout{LABEL\arabic{I}}%
    \edef\ddlabel{LABEL\arabic{I}}%
    \ddlabel: \DDboxAutoWhite[#1]{\ddlabel}{\usebox{\csname BOX\alph{I}\endcsname}}\par
    % \ddlabel: \DDboxGen[white]{\ddlabel}{7ex}{4ex}{\usebox{\csname BOX\alph{I}\endcsname}}\par
  }%
  BOTTOM \par
  \closeoutputstream{dnd}%
}

\newwrite\mywrite
\openout\mywrite=\jobname.pos\relax
\newif\ifddhidebox

% by default, all math inline; this can change that behavior
%\everymath{\displaystyle}

\newcommand{\DDboxGen}[5][blue]{%
    % generic drag-drop box
    %
    % #1 = optional color for box line
    % #2 = box target name
    % #3 = width
    % #4 = height
    % #5 = text to output inside box
    \typeout{Box #1: #2 by #3, text=#4}%
    \begin{tikzpicture}[color=#1, line width=0.25mm, baseline]
        \draw [dashed] (0,-0.25)
            node {\zsavepos{box#2-ll}}
            rectangle (#3,#4)
            node {\zsavepos{box#2-ur}};
        \draw (0,0)
            %node (EQ) [anchor=base] {#5};	% cuts off left side
            node (EQ) [anchor=base west] {#5};	% aligns to west baseline
    \write\mywrite{box#2: \zposx{box#2-ll}, \zposy{box#2-ll}, \zposx{box#2-ur}, \zposy{box#2-ur}}%
    \end{tikzpicture}%
}

\newcommand{\DDbox}[4]{%
    % drag-drop box, with same arguments as \DDboxGen, but #4 is a label name
    \addtostream{dnd}{BOX: #1 = #4}%
    \DDboxGen{#1}{#2}{#3}{\expandafter\usebox{\csname BOXName#4\endcsname}}%
}

\newlength\bwidth
\newlength\bheight

\newcommand{\DDboxAutoWhite}[3][auto]{%
    % drag-drop box, with auto-width
    % #1 = box height (optional)
    % #2 = box target name
    % #3 = box contents
    \setlength{\bwidth}{\widthof{#3}+1.8ex}%
    % \setlength{\bheight}{\totalheightof{#2}+1.5ex}%
    \ifthenelse{\equal{#1}{auto}}{
      \setlength{\bheight}{\heightof{#3}+1.75ex}%
    }{
      \setlength{\bheight}{#1}%
    }
    \DDboxGen[white]{#2}{\bwidth}{\bheight}{#3}%
    %\DDboxGen[green]{#2}{\bwidth}{\bheight}{#3}%  use this for debugging
}
 to include this at the top of your tex file
%

\usepackage{zref-user}
\usepackage{zref-abspos}
\usepackage{tikz}
\usepackage{forloop}
\usepackage{calc}
\usepackage{newfile}
\usepackage{ifthen}

\newoutputstream{dnd}
\openoutputfile{\jobname.dnd}{dnd}

\newcounter{BoxCount}
\setcounter{BoxCount}{0}
\newcounter{I}

\pagestyle{empty}
\usetikzlibrary{calc}

\newcommand\DDlabel[2]{%
  % #1 = label name
  % #2 = label contents, e.g. math symbols or word
  \stepcounter{BoxCount}%
  \expandafter\newsavebox\expandafter{\csname BOX\alph{BoxCount}\endcsname}%
  \expandafter\sbox\expandafter{\csname BOX\alph{BoxCount}\endcsname}{{#2}}%
  \expandafter\edef\csname BOXName#1\endcsname{\csname BOX\alph{BoxCount}\endcsname}%
  \addtostream{dnd}{LABEL: \arabic{BoxCount} = #1}%
}%

\newcommand{\writeDDlabels}[1][4.3ex]{
  % write out drag-drop labels
  % arguments:
  %
  % #1 = box height (optional, defaults to 4.3ex)
  \newpage
  TOP \par
  \stepcounter{BoxCount}%
  \setcounter{I}{1}%
  \forloop{I}{1}{\value{I} < \value{BoxCount}}{%
    \typeout{LABEL\arabic{I}}%
    \edef\ddlabel{LABEL\arabic{I}}%
    \ddlabel: \DDboxAutoWhite[#1]{\ddlabel}{\usebox{\csname BOX\alph{I}\endcsname}}\par
    % \ddlabel: \DDboxGen[white]{\ddlabel}{7ex}{4ex}{\usebox{\csname BOX\alph{I}\endcsname}}\par
  }%
  BOTTOM \par
  \closeoutputstream{dnd}%
}

\newwrite\mywrite
\openout\mywrite=\jobname.pos\relax
\newif\ifddhidebox

% by default, all math inline; this can change that behavior
%\everymath{\displaystyle}

\newcommand{\DDboxGen}[5][blue]{%
    % generic drag-drop box
    %
    % #1 = optional color for box line
    % #2 = box target name
    % #3 = width
    % #4 = height
    % #5 = text to output inside box
    \typeout{Box #1: #2 by #3, text=#4}%
    \begin{tikzpicture}[color=#1, line width=0.25mm, baseline]
        \draw [dashed] (0,-0.25)
            node {\zsavepos{box#2-ll}}
            rectangle (#3,#4)
            node {\zsavepos{box#2-ur}};
        \draw (0,0)
            %node (EQ) [anchor=base] {#5};	% cuts off left side
            node (EQ) [anchor=base west] {#5};	% aligns to west baseline
    \write\mywrite{box#2: \zposx{box#2-ll}, \zposy{box#2-ll}, \zposx{box#2-ur}, \zposy{box#2-ur}}%
    \end{tikzpicture}%
}

\newcommand{\DDbox}[4]{%
    % drag-drop box, with same arguments as \DDboxGen, but #4 is a label name
    \addtostream{dnd}{BOX: #1 = #4}%
    \DDboxGen{#1}{#2}{#3}{\expandafter\usebox{\csname BOXName#4\endcsname}}%
}

\newlength\bwidth
\newlength\bheight

\newcommand{\DDboxAutoWhite}[3][auto]{%
    % drag-drop box, with auto-width
    % #1 = box height (optional)
    % #2 = box target name
    % #3 = box contents
    \setlength{\bwidth}{\widthof{#3}+1.8ex}%
    % \setlength{\bheight}{\totalheightof{#2}+1.5ex}%
    \ifthenelse{\equal{#1}{auto}}{
      \setlength{\bheight}{\heightof{#3}+1.75ex}%
    }{
      \setlength{\bheight}{#1}%
    }
    \DDboxGen[white]{#2}{\bwidth}{\bheight}{#3}%
    %\DDboxGen[green]{#2}{\bwidth}{\bheight}{#3}%  use this for debugging
}
 to include this at the top of your tex file
%

\usepackage{zref-user}
\usepackage{zref-abspos}
\usepackage{tikz}
\usepackage{forloop}
\usepackage{calc}
\usepackage{newfile}
\usepackage{ifthen}

\newoutputstream{dnd}
\openoutputfile{\jobname.dnd}{dnd}

\newcounter{BoxCount}
\setcounter{BoxCount}{0}
\newcounter{I}

\pagestyle{empty}
\usetikzlibrary{calc}

\newcommand\DDlabel[2]{%
  % #1 = label name
  % #2 = label contents, e.g. math symbols or word
  \stepcounter{BoxCount}%
  \expandafter\newsavebox\expandafter{\csname BOX\alph{BoxCount}\endcsname}%
  \expandafter\sbox\expandafter{\csname BOX\alph{BoxCount}\endcsname}{{#2}}%
  \expandafter\edef\csname BOXName#1\endcsname{\csname BOX\alph{BoxCount}\endcsname}%
  \addtostream{dnd}{LABEL: \arabic{BoxCount} = #1}%
}%

\newcommand{\writeDDlabels}[1][4.3ex]{
  % write out drag-drop labels
  % arguments:
  %
  % #1 = box height (optional, defaults to 4.3ex)
  \newpage
  TOP \par
  \stepcounter{BoxCount}%
  \setcounter{I}{1}%
  \forloop{I}{1}{\value{I} < \value{BoxCount}}{%
    \typeout{LABEL\arabic{I}}%
    \edef\ddlabel{LABEL\arabic{I}}%
    \ddlabel: \DDboxAutoWhite[#1]{\ddlabel}{\usebox{\csname BOX\alph{I}\endcsname}}\par
    % \ddlabel: \DDboxGen[white]{\ddlabel}{7ex}{4ex}{\usebox{\csname BOX\alph{I}\endcsname}}\par
  }%
  BOTTOM \par
  \closeoutputstream{dnd}%
}

\newwrite\mywrite
\openout\mywrite=\jobname.pos\relax
\newif\ifddhidebox

% by default, all math inline; this can change that behavior
%\everymath{\displaystyle}

\newcommand{\DDboxGen}[5][blue]{%
    % generic drag-drop box
    %
    % #1 = optional color for box line
    % #2 = box target name
    % #3 = width
    % #4 = height
    % #5 = text to output inside box
    \typeout{Box #1: #2 by #3, text=#4}%
    \begin{tikzpicture}[color=#1, line width=0.25mm, baseline]
        \draw [dashed] (0,-0.25)
            node {\zsavepos{box#2-ll}}
            rectangle (#3,#4)
            node {\zsavepos{box#2-ur}};
        \draw (0,0)
            %node (EQ) [anchor=base] {#5};	% cuts off left side
            node (EQ) [anchor=base west] {#5};	% aligns to west baseline
    \write\mywrite{box#2: \zposx{box#2-ll}, \zposy{box#2-ll}, \zposx{box#2-ur}, \zposy{box#2-ur}}%
    \end{tikzpicture}%
}

\newcommand{\DDbox}[4]{%
    % drag-drop box, with same arguments as \DDboxGen, but #4 is a label name
    \addtostream{dnd}{BOX: #1 = #4}%
    \DDboxGen{#1}{#2}{#3}{\expandafter\usebox{\csname BOXName#4\endcsname}}%
}

\newlength\bwidth
\newlength\bheight

\newcommand{\DDboxAutoWhite}[3][auto]{%
    % drag-drop box, with auto-width
    % #1 = box height (optional)
    % #2 = box target name
    % #3 = box contents
    \setlength{\bwidth}{\widthof{#3}+1.8ex}%
    % \setlength{\bheight}{\totalheightof{#2}+1.5ex}%
    \ifthenelse{\equal{#1}{auto}}{
      \setlength{\bheight}{\heightof{#3}+1.75ex}%
    }{
      \setlength{\bheight}{#1}%
    }
    \DDboxGen[white]{#2}{\bwidth}{\bheight}{#3}%
    %\DDboxGen[green]{#2}{\bwidth}{\bheight}{#3}%  use this for debugging
}
 to include this at the top of your tex file
%

\usepackage{zref-user}
\usepackage{zref-abspos}
\usepackage{tikz}
\usepackage{forloop}
\usepackage{calc}
\usepackage{newfile}
\usepackage{ifthen}

\newoutputstream{dnd}
\openoutputfile{\jobname.dnd}{dnd}

\newcounter{BoxCount}
\setcounter{BoxCount}{0}
\newcounter{I}

\pagestyle{empty}
\usetikzlibrary{calc}

\newcommand\DDlabel[2]{%
  % #1 = label name
  % #2 = label contents, e.g. math symbols or word
  \stepcounter{BoxCount}%
  \expandafter\newsavebox\expandafter{\csname BOX\alph{BoxCount}\endcsname}%
  \expandafter\sbox\expandafter{\csname BOX\alph{BoxCount}\endcsname}{{#2}}%
  \expandafter\edef\csname BOXName#1\endcsname{\csname BOX\alph{BoxCount}\endcsname}%
  \addtostream{dnd}{LABEL: \arabic{BoxCount} = #1}%
}%

\newcommand{\writeDDlabels}[1][4.3ex]{
  % write out drag-drop labels
  % arguments:
  %
  % #1 = box height (optional, defaults to 4.3ex)
  \newpage
  TOP \par
  \stepcounter{BoxCount}%
  \setcounter{I}{1}%
  \forloop{I}{1}{\value{I} < \value{BoxCount}}{%
    \typeout{LABEL\arabic{I}}%
    \edef\ddlabel{LABEL\arabic{I}}%
    \ddlabel: \DDboxAutoWhite[#1]{\ddlabel}{\usebox{\csname BOX\alph{I}\endcsname}}\par
    % \ddlabel: \DDboxGen[white]{\ddlabel}{7ex}{4ex}{\usebox{\csname BOX\alph{I}\endcsname}}\par
  }%
  BOTTOM \par
  \closeoutputstream{dnd}%
}

\newwrite\mywrite
\openout\mywrite=\jobname.pos\relax
\newif\ifddhidebox

% by default, all math inline; this can change that behavior
%\everymath{\displaystyle}

\newcommand{\DDboxGen}[5][blue]{%
    % generic drag-drop box
    %
    % #1 = optional color for box line
    % #2 = box target name
    % #3 = width
    % #4 = height
    % #5 = text to output inside box
    \typeout{Box #1: #2 by #3, text=#4}%
    \begin{tikzpicture}[color=#1, line width=0.25mm, baseline]
        \draw [dashed] (-0.5,-0.25)
            node {\zsavepos{box#2-ll}}
            rectangle (#3,#4)
            node {\zsavepos{box#2-ur}};
        \draw (0,0)
            node (EQ) [anchor=base] {#5};
    \write\mywrite{box#2: \zposx{box#2-ll}, \zposy{box#2-ll}, \zposx{box#2-ur}, \zposy{box#2-ur}}%
    \end{tikzpicture}%
}

\newcommand{\DDbox}[4]{%
    % drag-drop box, with same arguments as \DDboxGen, but #4 is a label name
    \addtostream{dnd}{BOX: #1 = #4}%
    \DDboxGen{#1}{#2}{#3}{\expandafter\usebox{\csname BOXName#4\endcsname}}%
}

\newlength\bwidth
\newlength\bheight

\newcommand{\DDboxAutoWhite}[3][auto]{%
    % drag-drop box, with auto-width
    % #1 = box height (optional)
    % #2 = box target name
    % #3 = box contents
    \setlength{\bwidth}{\widthof{#3}+1.6ex}%
    % \setlength{\bheight}{\totalheightof{#2}+1.5ex}%
    \ifthenelse{\equal{#1}{auto}}{
      \setlength{\bheight}{\heightof{#3}+1.75ex}%
    }{
      \setlength{\bheight}{#1}%
    }
    \DDboxGen[white]{#2}{\bwidth}{\bheight}{#3}%
}
